\section{Key Types}

\subsection{LexBase}

\begin{code}
data Identifier = Id String Int
\end{code}

\begin{code}
data Token
 = TA String
 | TI Identifier
pattern ArbTok s = TA s
pattern IdTok i = TI i
\end{code}

\subsection{Variables}

\begin{code}
data VarClass -- Classification
  = VO -- Observation
  | VE -- Expression
  | VP -- Predicate
pattern ObsV  = VO
pattern ExprV = VE
pattern PredV = VP
\end{code}

\begin{code}
type Subscript = String
data VarWhen -- Variable role
  = WS            --  Static
  | WB            --  Before (pre)
  | WD Subscript  --  During (intermediate)
  | WA            --  After (post)
  | WT            --  Textual
pattern Static    =  WS
pattern Before    =  WB
pattern During n  =  WD n
pattern After     =  WA
pattern Textual   =  WT
\end{code}

\begin{code}
data GenVar
 = GV Variable -- regular variable
 | GL ListVar  -- variable denoting a list of variables
pattern StdVar v = GV v
pattern LstVar lv = GL lv
type VarList = [GenVar]
type VarSet = Set GenVar
\end{code}

\newpage
\subsection{AST}

\begin{code}
type TermSub = Set (Variable,Term) -- target variable, then replacememt term
type LVarSub = Set (ListVar,ListVar) -- target list-variable, then replacement l.v.
data Substn --  pair-sets below are unique in fst part
  = SN TermSub LVarSub
  deriving (Eq,Ord,Show,Read)
\end{code}

\begin{code}
data Type -- most general types first
 = T  -- arbitrary type
 | TV Identifier -- type variable
 | TC Identifier [Type] -- type constructor, applied
 | TA Identifier [(Identifier,[Type])] -- algebraic data type
 | TF Type Type -- function type
 | TG Identifier -- given type
pattern ArbType = T
pattern TypeVar i  = TV i
pattern TypeApp i ts = TC i ts
pattern DataType i fs = TA i fs
pattern FunType tf ta = TF tf ta
pattern GivenType i = TG i
\end{code}


\begin{code}
type Txt = String
data Value
 = VB Bool
 | VI Integer
 | VT Txt
pattern Boolean b  =  VB b
pattern Integer i  =  VI i
pattern Txt     t  =  VT t
\end{code}

\begin{code}
data TermKind
 = P -- predicate
 | E Type -- expression (with type annotation)
\end{code}

\begin{code}
data Term
 = K TermKind Value                    -- Value
 | V TermKind Variable                 -- Variable
 | C TermKind Identifier [Term]        -- Constructor
 | B TermKind Identifier VarSet Term   -- Binder (unordered)
 | L TermKind Identifier VarList Term  -- Binder (ordered)
 | X Identifier Term                   -- Closure (always a predicate)
 | S TermKind Term Substn              -- Substitution
 | I TermKind                          -- Iterator
     Identifier  -- top grouping constructor
     Identifier  -- component constructor, with arity a
     [ListVar]   -- list-variables, same length as component arity
 | ET Type                              -- Embedded TypeVar
\end{code}

\begin{code}
pattern Val  tk k          =   K tk k
pattern Var  tk v          <-  V tk v
pattern Cons tk n ts       =   C tk n ts
pattern Bnd  tk n vs tm    <-  B tk n vs tm
pattern Lam  tk n vl tm    <-  L tk n vl tm
pattern Cls     n    tm    =   X n tm
pattern Sub  tk tm s       =   S tk tm s
pattern Iter tk na ni lvs  =   I tk na ni lvs
pattern Typ  typ           =   ET typ
\end{code}

\begin{code}
pattern EVal t k           =  K (E t) k
pattern EVar t v          <-  V (E t) v
pattern ECons t n ts       =  C (E t) n ts
pattern EBind t n vs tm   <-  B (E t) n vs tm
pattern ELam t n vl tm    <-  L (E t) n vl tm
pattern ESub t tm s        =  S (E t) tm s
pattern EIter t na ni lvs  =  I (E t) na ni lvs
\end{code}

\begin{code}
pattern PVal k             =  K P k
pattern PVar v            <-  V P v
pattern PCons n ts         =  C P n ts
pattern PBind n vs tm     <-  B P n vs tm
pattern PLam n vl tm      <-  L P n vl tm
pattern PSub tm s          =  S P tm s
pattern PIter na ni lvs    =  I P na ni lvs
\end{code}

\begin{code}
pattern E2 t n t1 t2  = C (E t) n [t1,t2]
pattern P2   n t1 t2  = C P     n [t1,t2]
\end{code}

\subsection{VarData}

\begin{code}
data VarMatchRole -- Variable Matching Role
  =  KC Term     -- Known Constant
  |  KV Type     -- Known Variable
  |  KG          -- Generic Variable
  |  KI Variable -- Instance Variable ! variable must be known as generic
  |  UV          -- Unknown Variable
pattern KnownConst trm = KC trm
pattern KnownVar typ   = KV typ
pattern GenericVar     = KG
pattern InstanceVar v  = KI v
pattern UnknownVar     = UV
\end{code}

\begin{code}
data LstVarMatchRole -- ListVar Matching Roles
 = KL VarList        -- Known Variable-List, all of which are themselves known
      [Variable]     -- full expansion
      Int            -- length of full expansion
 | KS VarSet         -- Known Variable-Set, all of which are themselves known
      (Set Variable) -- full expansion
      Int            -- size of full expansion
 | AL                -- Abstract Known Variable-List
 | AS                -- Abstract Known Variable-Set
 | UL                -- Unknown List-Variable
pattern KnownVarList vl vars len  =  KL vl vars len
pattern KnownVarSet  vs vars siz  =  KS vs vars siz
pattern AbstractList              =  AL
pattern AbstractSet               =  AS
pattern UnknownListVar            =  UL
\end{code}

\subsection{SideCond}

\begin{code}
data AtmSideCond
 = SD  GenVar VarSet -- Disjoint
 | SS  GenVar VarSet -- Superset (covers)
 | SP  GenVar        -- Pre
pattern Disjoint gv vs = SD  gv vs  --  vs `intersect`  gv = {}
pattern Covers   gv vs = SS  gv vs  --  vs `supersetof` gv
pattern IsPre    gv    = SP  gv     --  gv is pre-condition
\end{code}

\begin{code}
type SideCond = ( [AtmSideCond]  -- all must be true
                , VarSet )       -- must be fresh
\end{code}

\begin{code}
type Assertion = (Term, SideCond)
\end{code}

\newpage
\subsection{Laws}

\begin{code}
data LogicSig
  = LogicSig
     { theTrue  :: Term
     , theFalse :: Term
     , theEqv   :: Identifier
     , theImp   :: Identifier
     , theAnd   :: Identifier
     , theOr    :: Identifier
     }
\end{code}

\begin{code}
data LeftRight = Lft | Rght deriving (Eq,Show,Read)

data GroupSpec
  = Assoc LeftRight
  | Gather LeftRight Int
  | Split Int
  deriving (Eq,Show,Read)
\end{code}

\begin{code}
type NmdAssertion = (String,Assertion)
\end{code}

\begin{code}
data Provenance
  = Axiom          --  considered as `self-evidently` True
  | Proven String  --  demonstrated by (named) proof
  | Assumed        --  conjecture asserted w/o proof
  deriving (Eq,Show,Read)
\end{code}

\begin{code}
type Law = (NmdAssertion,Provenance)
\end{code}

\newpage
\subsection{Proofs}

\begin{code}
data MatchClass
  = MA       -- match all of law, with replacement 'true'
  | ME [Int] -- match subpart of 'equiv' chain
  | MIA      -- match implication antecedent A, replacement A /\ C
  | MIC      -- match implication consequent C, replacement A \/ C
  -- MEV should be last, so these matches rank low by default
  | MEV Int  -- match PredVar at given position
pattern MatchAll       = MA
pattern MatchEqv is    = ME is
pattern MatchAnte      = MIA
pattern MatchCnsq      = MIC
pattern MatchEqvVar i  = MEV i
\end{code}

\begin{code}
data HowUsed
  = ByMatch MatchClass  -- replace focus with binding(match)
  | ByInstantiation     -- replace focus=true with binding(law)
  deriving (Eq,Show,Read)
\end{code}

\begin{code}
data SeqFocus = CLeft | CRight | Hyp Int deriving (Eq,Show,Read)
\end{code}

\begin{code}
data Justification
  = UseLaw             -- used a law
      HowUsed              -- how law was used in proof step
      String               -- law name
      Binding              -- binding from law variables to goal components
      [Int]                -- zipper descent arguments
  | Substitute         -- performed a substitution
      [Int]                -- zipper descent arguments
  | NormQuant          -- performed a quantifier normalisation
      [Int]                -- zipper descent arguments
  | NestSimp           -- simplified nested quantifiers
      [Int]                -- zipper descent arguments
  | Switch             -- switched focus at sequent level
      SeqFocus             -- focus before switch -- needed to reverse this.
      SeqFocus             -- focus after switch
  | CloneH Int         --  Cloned hypothesis i
  | Flatten Identifier -- flattened use of associative operator
  | Associate          -- grouped use of an associative operator
      Identifier           -- operator
      GroupSpec            -- grouping details.
  deriving (Eq,Show,Read)
\end{code}

\begin{code}
type CalcStep
  = ( Justification  -- step justification
    , Assertion )         -- previous term
\end{code}

\begin{code}
type Calculation
  = ( Term -- end (or current) term
    , [ CalcStep ] )  -- calculation steps, in proof order
\end{code}

\begin{code}
type Proof
  = ( String -- assertion name
    , Assertion
    , String -- Strategy
    , Calculation -- Simple calculational proofs for now
    )
\end{code}


\subsection{Theories}

\begin{code}
data Theory
  = Theory {
      thName   :: String
    , thDeps   :: [String]
    , known    :: VarTable
    , subable  :: SubAbilityMap
    , laws     :: [Law]
    , proofs   :: [Proof]
    , conjs    :: [NmdAssertion]
    }
\end{code}

\begin{code}
type TheoryMap = Map String Theory
data Theories
  = Theories { tmap :: TheoryMap
             , sdag :: SDAG String }
\end{code}

\subsection{Sequents}

\begin{code}
data Sequent
  = Sequent {
     ante :: [Theory] -- antecedent theory context
   , hyp :: Theory -- the goal hypotheses -- we can "go" here
   , sc :: SideCond -- of the conjecture being proven.
   , cleft :: Term -- never 'true' to begin with.
   , cright :: Term -- often 'true' from the start.
   }
  deriving (Eq, Show, Read)
\end{code}

\begin{code}
data Laws'
  = CLaws' { -- currently focussed on conjecture component
      hyp0  :: Theory -- hypothesis theory
    , whichC :: LeftRight -- which term is in the focus
    , otherC :: Term  -- the term not in the focus
    }
  | HLaws' { -- currently focussed on hypothesis component
      hname     :: String -- hyp. theory name
    , hknown    :: VarTable
    , hbefore   :: [Law] -- hyp. laws before focus (reversed)
    , fhName    :: String -- focus hypothesis name
    , fhSC      :: SideCond -- focus hypothesis sc (usually true)
    , fhProv    :: Provenance -- focus hypothesis provenance (?)
    , hOriginal :: Term -- the original form of the focus hypothesis
    , hafter    :: [Law] -- hyp. laws after focus
    , cleft0    :: Term -- left conjecture
    , cright0   :: Term -- right conjecture
    }
  deriving (Eq,Show,Read)
\end{code}

\begin{code}
data Sequent'
  = Sequent' {
      ante0 :: [Theory] -- context theories
    , sc0       :: SideCond -- sequent side-condition
    , laws'     :: Laws'
    }
  deriving (Eq,Show,Read)
\end{code}


\begin{code}
type SeqZip = (TermZip, Sequent')
\end{code}

\subsection{LiveProofs}

\begin{code}
data Match
 = MT { mName  ::  String     -- assertion name
      , mAsn   ::  Assertion  -- matched assertion
      , mClass ::  MatchClass -- match class
      , mBind  ::  Binding    -- resulting binding
      , mLocSC ::  SideCond   -- goal side-condition local update
      , mLawSC ::  SideCond   -- law side-condition mapped to goal
      , mRepl  ::  Term       -- replacement term
      } deriving (Eq,Show,Read)
\end{code}

\begin{code}
type MatchContext
  = ( String       -- Theory Name
    , [Law]        -- all laws of this theory
    , [VarTable] ) -- all known variables here, and in dependencies
\end{code}

\begin{code}
data LiveProof
  = LP {
      conjThName :: String -- conjecture theory name
    , conjName :: String -- conjecture name
    , conjecture :: Assertion -- assertion being proven
    , conjSC :: SideCond -- side condition
    , strategy :: String -- strategy
    , mtchCtxts :: [MatchContext] -- current matching contexts
    , focus :: SeqZip  -- current sub-term of interest
    , fPath :: [Int] -- current term zipper descent arguments
    , matches :: Matches -- current matches
    , stepsSoFar :: [CalcStep]  -- calculation steps so far, most recent first
    }
\end{code}

\begin{code}
type LiveProofs = Map (String,String) LiveProof
\end{code}
