\chapter{Classifying Laws}

In order to automate proof, 
we need a way to classify laws to help guide the tools.
We cannot rely on exhaustive search/try techniques
because we very quickly get into a situation that results in a 
combinatorial explosion.
We start with a simple classification that defines three classes:

\begin{description}
    \item[Simplify]
      are laws that when used reduce the size of a term.
    \item[Unfold/Expand]
      are laws that replace terms,
      that apply a defined name to argument terms,
      with the expansion of its definition,
      using those arguments.
    \item[Fold/Contract]
      are laws that replace terms
      that are the expansion of some name's definition,
      by a term that applies that name to appropriate argument terms.
\end{description}

We look at these cases in the context of equational reasoning,
which allows us to use laws of a certain form in multiple ways.
If any law matches a term in its entirety, 
then the replacement term is simply \true.
If a law has the form $P \equiv Q$ (or $e = f$),
then if one side of the equivalence (equality) matches a term,
we can replace that term by the other side, 
using the match binding to do substitution.
We refer to such laws as being \emph{equational}.

\subsubsection*{Simplify}

\emph{Any} law, applied in its \emph{entirety}, is a simplfication law
(term \true\ is simpler than any other term, except \false).
This is not as helpful as it sounds as indiscriminate use of laws
in this way leads to the aforementioned combinatorial explosion.
What is more interesting is to consider equational laws,
of the form $P \equiv Q$.
Such a law simplifies left-to-right if term $Q$ is smaller than term $P$,
and conversely, right-to-left if $P$ is smaller than $Q$.
If the sizes of $P$ and $Q$ are the same,
then neither partial use of that law is a simplification.
We compute the size of a term,
as the total number of names/atomic values it contains.
In quantified formula ($\forall ...$, $\exists ...$) 
we count those two symbols as names worth one%
\footnote{
Often eliminating quantifiers is a sign of progress, 
so another option would be give them a higher weighting for term size.
}%
.

\subsubsection*{Unfold/Expand}

All definitional laws, for our purposes, have to be equational.
By convention, the name being defined should be on the lefthand side,
and we shall assume this is the case here.
A definition that is the wrong way around can be reversed because both
equivalence and equality are symmetric.
Consider a law $P \equiv Q$. What makes it a definition of a name $N$?
The answer is, 
at the top-level,
that $P$ has the form of $N$ appled to zero or more terms, 
so the law has the form
$N(t_1,\dots,t_n) \equiv Q, \qquad n \geq 0$.
In this case, we say that applying this law left-to-right
is an unfold or expansion.

\subsubsection*{Fold/Contract}

Given the law $N(t_1,\dots,t_n) \equiv Q$ from above,
then applying it right-to-left is a fold or contraction.