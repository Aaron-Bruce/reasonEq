\section{Matching Features}\label{sec:match-feat}

In order to support alphabet-based laws and definitions,
we introduce the notion of ``list variables'',
that match against zero or more variables satisfying some property.
These list-variables encode simple expressions that give
some flexibility in how they match.
We further classify as follows:
\begin{description}
  \item[Reserved:]  $O$, $M$, $S$
  \item[Generic:] $\lst v$, $\lst e$, \ldots (lowercase)
\end{description}
Here are some examples of laws that require this list-matching facility:
\begin{description}
%
\item[Sequential Composition]
$$
  P ; Q ~\defs~ \exists O_m \bullet P[O_m/O'] \land Q[O_m/O]
$$
Here the $O$ list-variable matches all the undecorated
observation variables., i.e. the alphabet, of the theory.
In practise we also need a side condition that says all the $O_m$
are fresh --- we don't want to rely on $P$ and $Q$ only having free variables
in that alphabet.
The idea is that \texttt{O}, \texttt{M} and \texttt{S} are reserved-variable root strings that
correspond to $O$, $M$ and $S$, respectively.
%
\item[Assignment]
$$
  x := e ~\defs~ ok \implies ok' \land x' = e
    \land S'\less x = S\less x
$$
List-variable $S$ refers to all the program or user variables,
those that appear explicitly in the programming notation (script) under study.
This are different from the model variables (often referred to as
``auxiliary''), denoted by $M$ that capture observable aspects
of system behaviour beyond what can be expressed with just the $S$
values. Examples of variables in $M$ include $ok$, $wait$, $tr$, etc.
The expression $S'\less x = S\less x$
denotes the conjunction of identities of the form $v'=v$
where $v \in S\setminus\setof x$.
If $x$ is an observation variable
then the above law only works for that variable,
and in fact could be expanded out (given $S$)
to an explicit law for that variable.
If $x$ is not an observation variable,
then it should be a ``variable-variable'',
and typically we cannot complete the match of $S\less x$
until
other context information tells us to which variables $x$ must match.
\par
We can have more than one subtracted variable:
$$
  x,y := e,f ~\defs~ ok \implies ok' \land x' = e \land y'=f
    \land S'\less{x,y} = S\less{x,y}
$$
\item[Simultaneous Assignment]
Way more than just two subtracted variables!
\begin{eqnarray*}
  \lefteqn{x_1,\ldots,x_n ::= e_1, \ldots , e_n}
\\ &\defs&
     ok \implies ok'
     \land x'_1 = e_1 \land \ldots \land x'_n = e_n
     \land S'\less{x_1,\ldots,x_n} = S\less{x_1,\ldots,x_n}
\\ &=& ok \implies ok'
    \land \lst x := \lst e \land
     \land S'\less{\lst x} = S\less{\lst x}
\end{eqnarray*}
In the mathematical form, we use indices and ellipsis
to suggest that we have lists of variables and expressions
of the same length.
\item[Skip]
\begin{eqnarray*}
  \Skip &\defs& ok \implies ok' \land S'=S
\end{eqnarray*}
A simple form of a simple simultaneous assignment!
\end{description}

From the above examples,
a number of general observations can be made:
\begin{itemize}
  \item
    All list-variables match against lists of things (zero or more).
    This means they can only occur in certain places:
    \begin{itemize}
      \item quantifier binding variable lists
      \item replacement/target lists in substitutions
       \item in 2-place (atomic) predicates, provided both places
        are occupied by a single meta-variable.
    \end{itemize}
  \item
    A predicate involving two list-variables mandates that both
    match lists of the same size, and the predicate
    is interpreted as the \emph{conjunction} of instances of that predicate
    over corresponding pairs from the match-list.
    It can also match a single such predicate also using two list-variables.
  \item
    Some list-variables ($O$, $S$, $M$)
    denote specific sets of variables.
    The matching rules for these differ from those for regular variables,
    in that $O$ and $O'$ can only match themselves or their expansions,
    while $O_a$ and $O_b$ can match with a binding---subscript decorations
    are viewed as general patterns, and are not considered ``known''.
  \item
    Other list-variables  ($x_1$, $e_1$, \ldots)
    are local names used as general placeholder.
    These simply match all the variables/ expressions /whatever
    that occur in the relevant position.
  \item
    These list-variables won't only appear in match patterns,
    but could also appear in test/goal predicates,
    so we could imagine the following predicate fragment:
    $$
       \exists ok_m,S_m \bullet \ldots
    $$
    matching against a similar fragment from the definition of
    sequential composition:
    $$
      \exists O_n \bullet \ldots
    $$
    Here the match binding would include
    $O_n \mapsto \setof{ok_m} \cup S_m$.
  \item
    The reserved list-variables are \emph{precise},
    in the sense that on any given matching context,
    they match a known fixed collection of variables
    (so there shouldn't be a need in most cases to defer a match).
    An exception is in a generic theory that has no specific observation
    variables defined, when these list-variables can only match
    themselves, and $O$ being able to match $S,M$.
\end{itemize}
